% Generated by Sphinx.
\def\sphinxdocclass{report}
\documentclass[letterpaper,10pt,english]{sphinxmanual}
\usepackage[utf8]{inputenc}
\DeclareUnicodeCharacter{00A0}{\nobreakspace}
\usepackage{cmap}
\usepackage[T1]{fontenc}
\usepackage{babel}
\usepackage{times}
\usepackage[Bjarne]{fncychap}
\usepackage{longtable}
\usepackage{sphinx}
\usepackage{multirow}
\usepackage{eqparbox}


\addto\captionsenglish{\renewcommand{\figurename}{Fig. }}
\addto\captionsenglish{\renewcommand{\tablename}{Table }}
\SetupFloatingEnvironment{literal-block}{name=Listing }



\title{Timbre Explorer Documentation}
\date{October 22, 2016}
\release{0.0.1}
\author{Doug Olson}
\newcommand{\sphinxlogo}{}
\renewcommand{\releasename}{Release}
\setcounter{tocdepth}{3}
\makeindex

\makeatletter
\def\PYG@reset{\let\PYG@it=\relax \let\PYG@bf=\relax%
    \let\PYG@ul=\relax \let\PYG@tc=\relax%
    \let\PYG@bc=\relax \let\PYG@ff=\relax}
\def\PYG@tok#1{\csname PYG@tok@#1\endcsname}
\def\PYG@toks#1+{\ifx\relax#1\empty\else%
    \PYG@tok{#1}\expandafter\PYG@toks\fi}
\def\PYG@do#1{\PYG@bc{\PYG@tc{\PYG@ul{%
    \PYG@it{\PYG@bf{\PYG@ff{#1}}}}}}}
\def\PYG#1#2{\PYG@reset\PYG@toks#1+\relax+\PYG@do{#2}}

\expandafter\def\csname PYG@tok@gd\endcsname{\def\PYG@tc##1{\textcolor[rgb]{0.63,0.00,0.00}{##1}}}
\expandafter\def\csname PYG@tok@gu\endcsname{\let\PYG@bf=\textbf\def\PYG@tc##1{\textcolor[rgb]{0.50,0.00,0.50}{##1}}}
\expandafter\def\csname PYG@tok@gt\endcsname{\def\PYG@tc##1{\textcolor[rgb]{0.00,0.27,0.87}{##1}}}
\expandafter\def\csname PYG@tok@gs\endcsname{\let\PYG@bf=\textbf}
\expandafter\def\csname PYG@tok@gr\endcsname{\def\PYG@tc##1{\textcolor[rgb]{1.00,0.00,0.00}{##1}}}
\expandafter\def\csname PYG@tok@cm\endcsname{\let\PYG@it=\textit\def\PYG@tc##1{\textcolor[rgb]{0.25,0.50,0.56}{##1}}}
\expandafter\def\csname PYG@tok@vg\endcsname{\def\PYG@tc##1{\textcolor[rgb]{0.73,0.38,0.84}{##1}}}
\expandafter\def\csname PYG@tok@vi\endcsname{\def\PYG@tc##1{\textcolor[rgb]{0.73,0.38,0.84}{##1}}}
\expandafter\def\csname PYG@tok@mh\endcsname{\def\PYG@tc##1{\textcolor[rgb]{0.13,0.50,0.31}{##1}}}
\expandafter\def\csname PYG@tok@cs\endcsname{\def\PYG@tc##1{\textcolor[rgb]{0.25,0.50,0.56}{##1}}\def\PYG@bc##1{\setlength{\fboxsep}{0pt}\colorbox[rgb]{1.00,0.94,0.94}{\strut ##1}}}
\expandafter\def\csname PYG@tok@ge\endcsname{\let\PYG@it=\textit}
\expandafter\def\csname PYG@tok@vc\endcsname{\def\PYG@tc##1{\textcolor[rgb]{0.73,0.38,0.84}{##1}}}
\expandafter\def\csname PYG@tok@il\endcsname{\def\PYG@tc##1{\textcolor[rgb]{0.13,0.50,0.31}{##1}}}
\expandafter\def\csname PYG@tok@go\endcsname{\def\PYG@tc##1{\textcolor[rgb]{0.20,0.20,0.20}{##1}}}
\expandafter\def\csname PYG@tok@cp\endcsname{\def\PYG@tc##1{\textcolor[rgb]{0.00,0.44,0.13}{##1}}}
\expandafter\def\csname PYG@tok@gi\endcsname{\def\PYG@tc##1{\textcolor[rgb]{0.00,0.63,0.00}{##1}}}
\expandafter\def\csname PYG@tok@gh\endcsname{\let\PYG@bf=\textbf\def\PYG@tc##1{\textcolor[rgb]{0.00,0.00,0.50}{##1}}}
\expandafter\def\csname PYG@tok@ni\endcsname{\let\PYG@bf=\textbf\def\PYG@tc##1{\textcolor[rgb]{0.84,0.33,0.22}{##1}}}
\expandafter\def\csname PYG@tok@nl\endcsname{\let\PYG@bf=\textbf\def\PYG@tc##1{\textcolor[rgb]{0.00,0.13,0.44}{##1}}}
\expandafter\def\csname PYG@tok@nn\endcsname{\let\PYG@bf=\textbf\def\PYG@tc##1{\textcolor[rgb]{0.05,0.52,0.71}{##1}}}
\expandafter\def\csname PYG@tok@no\endcsname{\def\PYG@tc##1{\textcolor[rgb]{0.38,0.68,0.84}{##1}}}
\expandafter\def\csname PYG@tok@na\endcsname{\def\PYG@tc##1{\textcolor[rgb]{0.25,0.44,0.63}{##1}}}
\expandafter\def\csname PYG@tok@nb\endcsname{\def\PYG@tc##1{\textcolor[rgb]{0.00,0.44,0.13}{##1}}}
\expandafter\def\csname PYG@tok@nc\endcsname{\let\PYG@bf=\textbf\def\PYG@tc##1{\textcolor[rgb]{0.05,0.52,0.71}{##1}}}
\expandafter\def\csname PYG@tok@nd\endcsname{\let\PYG@bf=\textbf\def\PYG@tc##1{\textcolor[rgb]{0.33,0.33,0.33}{##1}}}
\expandafter\def\csname PYG@tok@ne\endcsname{\def\PYG@tc##1{\textcolor[rgb]{0.00,0.44,0.13}{##1}}}
\expandafter\def\csname PYG@tok@nf\endcsname{\def\PYG@tc##1{\textcolor[rgb]{0.02,0.16,0.49}{##1}}}
\expandafter\def\csname PYG@tok@si\endcsname{\let\PYG@it=\textit\def\PYG@tc##1{\textcolor[rgb]{0.44,0.63,0.82}{##1}}}
\expandafter\def\csname PYG@tok@s2\endcsname{\def\PYG@tc##1{\textcolor[rgb]{0.25,0.44,0.63}{##1}}}
\expandafter\def\csname PYG@tok@nt\endcsname{\let\PYG@bf=\textbf\def\PYG@tc##1{\textcolor[rgb]{0.02,0.16,0.45}{##1}}}
\expandafter\def\csname PYG@tok@nv\endcsname{\def\PYG@tc##1{\textcolor[rgb]{0.73,0.38,0.84}{##1}}}
\expandafter\def\csname PYG@tok@s1\endcsname{\def\PYG@tc##1{\textcolor[rgb]{0.25,0.44,0.63}{##1}}}
\expandafter\def\csname PYG@tok@ch\endcsname{\let\PYG@it=\textit\def\PYG@tc##1{\textcolor[rgb]{0.25,0.50,0.56}{##1}}}
\expandafter\def\csname PYG@tok@m\endcsname{\def\PYG@tc##1{\textcolor[rgb]{0.13,0.50,0.31}{##1}}}
\expandafter\def\csname PYG@tok@gp\endcsname{\let\PYG@bf=\textbf\def\PYG@tc##1{\textcolor[rgb]{0.78,0.36,0.04}{##1}}}
\expandafter\def\csname PYG@tok@sh\endcsname{\def\PYG@tc##1{\textcolor[rgb]{0.25,0.44,0.63}{##1}}}
\expandafter\def\csname PYG@tok@ow\endcsname{\let\PYG@bf=\textbf\def\PYG@tc##1{\textcolor[rgb]{0.00,0.44,0.13}{##1}}}
\expandafter\def\csname PYG@tok@sx\endcsname{\def\PYG@tc##1{\textcolor[rgb]{0.78,0.36,0.04}{##1}}}
\expandafter\def\csname PYG@tok@bp\endcsname{\def\PYG@tc##1{\textcolor[rgb]{0.00,0.44,0.13}{##1}}}
\expandafter\def\csname PYG@tok@c1\endcsname{\let\PYG@it=\textit\def\PYG@tc##1{\textcolor[rgb]{0.25,0.50,0.56}{##1}}}
\expandafter\def\csname PYG@tok@o\endcsname{\def\PYG@tc##1{\textcolor[rgb]{0.40,0.40,0.40}{##1}}}
\expandafter\def\csname PYG@tok@kc\endcsname{\let\PYG@bf=\textbf\def\PYG@tc##1{\textcolor[rgb]{0.00,0.44,0.13}{##1}}}
\expandafter\def\csname PYG@tok@c\endcsname{\let\PYG@it=\textit\def\PYG@tc##1{\textcolor[rgb]{0.25,0.50,0.56}{##1}}}
\expandafter\def\csname PYG@tok@mf\endcsname{\def\PYG@tc##1{\textcolor[rgb]{0.13,0.50,0.31}{##1}}}
\expandafter\def\csname PYG@tok@err\endcsname{\def\PYG@bc##1{\setlength{\fboxsep}{0pt}\fcolorbox[rgb]{1.00,0.00,0.00}{1,1,1}{\strut ##1}}}
\expandafter\def\csname PYG@tok@mb\endcsname{\def\PYG@tc##1{\textcolor[rgb]{0.13,0.50,0.31}{##1}}}
\expandafter\def\csname PYG@tok@ss\endcsname{\def\PYG@tc##1{\textcolor[rgb]{0.32,0.47,0.09}{##1}}}
\expandafter\def\csname PYG@tok@sr\endcsname{\def\PYG@tc##1{\textcolor[rgb]{0.14,0.33,0.53}{##1}}}
\expandafter\def\csname PYG@tok@mo\endcsname{\def\PYG@tc##1{\textcolor[rgb]{0.13,0.50,0.31}{##1}}}
\expandafter\def\csname PYG@tok@kd\endcsname{\let\PYG@bf=\textbf\def\PYG@tc##1{\textcolor[rgb]{0.00,0.44,0.13}{##1}}}
\expandafter\def\csname PYG@tok@mi\endcsname{\def\PYG@tc##1{\textcolor[rgb]{0.13,0.50,0.31}{##1}}}
\expandafter\def\csname PYG@tok@kn\endcsname{\let\PYG@bf=\textbf\def\PYG@tc##1{\textcolor[rgb]{0.00,0.44,0.13}{##1}}}
\expandafter\def\csname PYG@tok@cpf\endcsname{\let\PYG@it=\textit\def\PYG@tc##1{\textcolor[rgb]{0.25,0.50,0.56}{##1}}}
\expandafter\def\csname PYG@tok@kr\endcsname{\let\PYG@bf=\textbf\def\PYG@tc##1{\textcolor[rgb]{0.00,0.44,0.13}{##1}}}
\expandafter\def\csname PYG@tok@s\endcsname{\def\PYG@tc##1{\textcolor[rgb]{0.25,0.44,0.63}{##1}}}
\expandafter\def\csname PYG@tok@kp\endcsname{\def\PYG@tc##1{\textcolor[rgb]{0.00,0.44,0.13}{##1}}}
\expandafter\def\csname PYG@tok@w\endcsname{\def\PYG@tc##1{\textcolor[rgb]{0.73,0.73,0.73}{##1}}}
\expandafter\def\csname PYG@tok@kt\endcsname{\def\PYG@tc##1{\textcolor[rgb]{0.56,0.13,0.00}{##1}}}
\expandafter\def\csname PYG@tok@sc\endcsname{\def\PYG@tc##1{\textcolor[rgb]{0.25,0.44,0.63}{##1}}}
\expandafter\def\csname PYG@tok@sb\endcsname{\def\PYG@tc##1{\textcolor[rgb]{0.25,0.44,0.63}{##1}}}
\expandafter\def\csname PYG@tok@k\endcsname{\let\PYG@bf=\textbf\def\PYG@tc##1{\textcolor[rgb]{0.00,0.44,0.13}{##1}}}
\expandafter\def\csname PYG@tok@se\endcsname{\let\PYG@bf=\textbf\def\PYG@tc##1{\textcolor[rgb]{0.25,0.44,0.63}{##1}}}
\expandafter\def\csname PYG@tok@sd\endcsname{\let\PYG@it=\textit\def\PYG@tc##1{\textcolor[rgb]{0.25,0.44,0.63}{##1}}}

\def\PYGZbs{\char`\\}
\def\PYGZus{\char`\_}
\def\PYGZob{\char`\{}
\def\PYGZcb{\char`\}}
\def\PYGZca{\char`\^}
\def\PYGZam{\char`\&}
\def\PYGZlt{\char`\<}
\def\PYGZgt{\char`\>}
\def\PYGZsh{\char`\#}
\def\PYGZpc{\char`\%}
\def\PYGZdl{\char`\$}
\def\PYGZhy{\char`\-}
\def\PYGZsq{\char`\'}
\def\PYGZdq{\char`\"}
\def\PYGZti{\char`\~}
% for compatibility with earlier versions
\def\PYGZat{@}
\def\PYGZlb{[}
\def\PYGZrb{]}
\makeatother

\renewcommand\PYGZsq{\textquotesingle}

\begin{document}

\maketitle
\tableofcontents
\phantomsection\label{index::doc}



\chapter{Overview}
\label{index:overview}\label{index:timbre-explorer}\label{index:module-Timbre}\index{Timbre (module)}
The Timbre Explorer module creates Timbre objects for the investigation of the dissonance and consonance properties of musical timbres. The disMeasure() function
is a Python translation of \href{http://sethares.engr.wisc.edu/comprog.html}{William Sethares's matlab and C code}. I have added timbre objects and
functions to do plots of various kinds and to generate .wav files so you can hear the timbres and their dissonance / consonance patterns. Requires Matplotlib, Numpy and Scipy.
\begin{description}
\item[{Objects can be initialized as:}] \leavevmode\begin{itemize}
\item {} 
Even

\item {} 
Odd

\item {} 
Evenodd

\item {} 
Square

\item {} 
Sawtooth

\item {} 
Triangle

\item {} 
Beam

\item {} 
Custom

\end{itemize}

\end{description}

Most objects have the following properties:
\begin{itemize}
\item {} 
\textbf{f\_0}: the fundamental frequency. Default is 220 Hz

\item {} 
\textbf{n\_partials}: the desired number of partials in the tone. Default is 7

\item {} 
\textbf{attenuation}: the attenuation rate to be applied to the partials. Default is .7071

\end{itemize}

Specific waveforms such as Square, Sawtooth etc, have predefined spectra and attenuation rates

The Custom object allows the user to specify any desired set of partials and amplitudes.

Typical usage, assuming you cd to the directory that contains the \textbf{Timbre} directory and run Python 2.7x from there:

\begin{Verbatim}[commandchars=\\\{\}]
\PYG{g+gp}{\PYGZgt{}\PYGZgt{}\PYGZgt{} }\PYG{k+kn}{import} \PYG{n+nn}{Timbre}
\PYG{g+gp}{\PYGZgt{}\PYGZgt{}\PYGZgt{} }\PYG{n}{foo} \PYG{o}{=} \PYG{n}{Timbre}\PYG{o}{.}\PYG{n}{Even}\PYG{p}{(}\PYG{p}{)} \PYG{c+c1}{\PYGZsh{} Create an object with even partials and default parameters}
\PYG{g+gp}{\PYGZgt{}\PYGZgt{}\PYGZgt{} }\PYG{n}{bar} \PYG{o}{=} \PYG{n}{Timbre}\PYG{o}{.}\PYG{n}{Beam}\PYG{p}{(}\PYG{n}{f\PYGZus{}0} \PYG{o}{=} \PYG{l+m+mi}{327}\PYG{p}{,} \PYG{n}{numPartials} \PYG{o}{=} \PYG{l+m+mi}{17}\PYG{p}{,} \PYG{n}{attenuation} \PYG{o}{=} \PYG{o}{.}\PYG{l+m+mi}{5}\PYG{p}{)} \PYG{c+c1}{\PYGZsh{} Create an object with partials of a vibrating beam}
\PYG{g+gp}{\PYGZgt{}\PYGZgt{}\PYGZgt{} }\PYG{n}{freqs} \PYG{o}{=} \PYG{p}{[}\PYG{l+m+mi}{300}\PYG{p}{,} \PYG{l+m+mi}{600}\PYG{p}{,} \PYG{l+m+mi}{900}\PYG{p}{,} \PYG{l+m+mi}{1200}\PYG{p}{]}
\PYG{g+gp}{\PYGZgt{}\PYGZgt{}\PYGZgt{} }\PYG{n}{amps} \PYG{o}{=} \PYG{p}{[}\PYG{l+m+mi}{1}\PYG{p}{,} \PYG{o}{.}\PYG{l+m+mi}{5}\PYG{p}{,} \PYG{o}{.}\PYG{l+m+mi}{3}\PYG{p}{,} \PYG{o}{.}\PYG{l+m+mi}{9}\PYG{p}{]}
\PYG{g+gp}{\PYGZgt{}\PYGZgt{}\PYGZgt{} }\PYG{n}{baz} \PYG{o}{=} \PYG{n}{Timbre}\PYG{o}{.}\PYG{n}{Custom}\PYG{p}{(}\PYG{n}{freqs}\PYG{p}{,} \PYG{n}{amps}\PYG{p}{)} \PYG{c+c1}{\PYGZsh{} Create an object with custom partials}
\PYG{g+gp}{\PYGZgt{}\PYGZgt{}\PYGZgt{} }\PYG{n}{foo}\PYG{o}{.}\PYG{n}{disPlot}\PYG{p}{(}\PYG{p}{)} \PYG{c+c1}{\PYGZsh{} plot the dissonance curve for the timbre }
\PYG{g+gp}{\PYGZgt{}\PYGZgt{}\PYGZgt{} }\PYG{n}{bar}\PYG{o}{.}\PYG{n}{ConsDisFreqs}\PYG{p}{(}\PYG{n}{makePlot} \PYG{o}{=} \PYG{n+nb+bp}{True}\PYG{p}{)} \PYG{c+c1}{\PYGZsh{} plot identifies maximima and minima in the dissonance curve}
\PYG{g+gp}{\PYGZgt{}\PYGZgt{}\PYGZgt{} }\PYG{n}{baz}\PYG{o}{.}\PYG{n}{partialsPlot}\PYG{p}{(}\PYG{p}{)} \PYG{c+c1}{\PYGZsh{} bar plot of the relative amplitudes and frequencies for the partials of the timbre }
\PYG{g+gp}{\PYGZgt{}\PYGZgt{}\PYGZgt{} }\PYG{n}{foo}\PYG{o}{.}\PYG{n}{wavePlot}\PYG{p}{(}\PYG{p}{)} \PYG{c+c1}{\PYGZsh{} plot one period of the timbre\PYGZsq{}s waveform}
\PYG{g+gp}{\PYGZgt{}\PYGZgt{}\PYGZgt{} }\PYG{n}{bar}\PYG{o}{.}\PYG{n}{timbreGen}\PYG{p}{(}\PYG{p}{)} \PYG{c+c1}{\PYGZsh{} Generate a 5 second sample of the timbre }
\PYG{g+gp}{\PYGZgt{}\PYGZgt{}\PYGZgt{} }\PYG{n}{baz}\PYG{o}{.}\PYG{n}{timbreSweep}\PYG{p}{(}\PYG{n}{length} \PYG{o}{=} \PYG{l+m+mi}{60}\PYG{p}{)} \PYG{c+c1}{\PYGZsh{} Generate a sweep of the timbre against itself}
\PYG{g+gp}{\PYGZgt{}\PYGZgt{}\PYGZgt{} }\PYG{n}{Timbre}\PYG{o}{.}\PYG{n}{disPlotMultiple}\PYG{p}{(}\PYG{n}{foo}\PYG{p}{,} \PYG{n}{bar}\PYG{p}{,} \PYG{n}{baz}\PYG{p}{)} \PYG{c+c1}{\PYGZsh{} Generate a dissonance plot for timbres foo, bar and baz}
\PYG{g+gp}{\PYGZgt{}\PYGZgt{}\PYGZgt{} }\PYG{n}{foo}\PYG{o}{.}\PYG{n}{writeConsonantChord}\PYG{p}{(}\PYG{p}{)} \PYG{c+c1}{\PYGZsh{} Writes .wav chord, user selected intervals, based on consonant frequencies}
\PYG{g+gp}{\PYGZgt{}\PYGZgt{}\PYGZgt{} }\PYG{n}{foo}\PYG{o}{.}\PYG{n}{writeEqTempChord}\PYG{p}{(}\PYG{p}{)} \PYG{c+c1}{\PYGZsh{} Writes .wav chord, user selected intervals, based on equal tempered frequencies}
\end{Verbatim}


\chapter{Timbre.timbre module}
\label{index:module-Timbre.timbre}\label{index:timbre-timbre-module}\index{Timbre.timbre (module)}\index{Timbre (class in Timbre.timbre)}

\begin{fulllineitems}
\phantomsection\label{index:Timbre.timbre.Timbre}\pysiglinewithargsret{\strong{class }\code{Timbre.timbre.}\bfcode{Timbre}}{\emph{f\_0}, \emph{numPartials}, \emph{octaves=1}}{}
Bases: \code{object}

Parent class for the various timbre objects. Use subclasses Even, Odd, Evenodd, Square, Sawtooth, Triangle, Beam and Custom to create objects
\index{audioGenPath() (Timbre.timbre.Timbre method)}

\begin{fulllineitems}
\phantomsection\label{index:Timbre.timbre.Timbre.audioGenPath}\pysiglinewithargsret{\bfcode{audioGenPath}}{}{}
Creates a directory for audio files to be written to.

\end{fulllineitems}

\index{consDisFreqs() (Timbre.timbre.Timbre method)}

\begin{fulllineitems}
\phantomsection\label{index:Timbre.timbre.Timbre.consDisFreqs}\pysiglinewithargsret{\bfcode{consDisFreqs}}{\emph{makePlot=False}}{}
Generates a list of the consonances (minima on the dissonance plot) for a given timbre. The consonances are used by consFreqs and writeChord. Optionally plots a bar graph                  showing the peak dissonant an consonant frequencies for a given Timbre object

\end{fulllineitems}

\index{consFreqs() (Timbre.timbre.Timbre method)}

\begin{fulllineitems}
\phantomsection\label{index:Timbre.timbre.Timbre.consFreqs}\pysiglinewithargsret{\bfcode{consFreqs}}{\emph{makePlot=False}}{}
Generates an array of consonant frequencies for a given Timbre. Optionally plots a bar graph showing the peak dissonant an consonant frequencies for a given Timbre object

\end{fulllineitems}

\index{disMeasure() (Timbre.timbre.Timbre method)}

\begin{fulllineitems}
\phantomsection\label{index:Timbre.timbre.Timbre.disMeasure}\pysiglinewithargsret{\bfcode{disMeasure}}{\emph{octaves=1}}{}
Calculates the relative dissonance for a given Timbre object at all intervals within a specified range. disMeasure(octaves = 1).

returns an an array of relative dissonances \textbf{self.dissonances} and a normalized dissonance array \textbf{self.norm}

\end{fulllineitems}

\index{disPlot() (Timbre.timbre.Timbre method)}

\begin{fulllineitems}
\phantomsection\label{index:Timbre.timbre.Timbre.disPlot}\pysiglinewithargsret{\bfcode{disPlot}}{\emph{normalized=True}}{}
Plots relative dissonance at all intervals for a given tone

\end{fulllineitems}

\index{partialsPlot() (Timbre.timbre.Timbre method)}

\begin{fulllineitems}
\phantomsection\label{index:Timbre.timbre.Timbre.partialsPlot}\pysiglinewithargsret{\bfcode{partialsPlot}}{}{}
Plots the partials and their relative amplitudes for the timbre being examined. Positive Amplitudes in red, Negative amplitudes in blue.

\end{fulllineitems}

\index{timbreGen() (Timbre.timbre.Timbre method)}

\begin{fulllineitems}
\phantomsection\label{index:Timbre.timbre.Timbre.timbreGen}\pysiglinewithargsret{\bfcode{timbreGen}}{}{}
Generates audio data and a .wav file of a timbre. 
* prompts user for file length
* files in TimbreAudio in cwd

\end{fulllineitems}

\index{timbreSweep() (Timbre.timbre.Timbre method)}

\begin{fulllineitems}
\phantomsection\label{index:Timbre.timbre.Timbre.timbreSweep}\pysiglinewithargsret{\bfcode{timbreSweep}}{}{}
Generates audio data and a .wav file of a timbre swept against itself, i.e. one tone is held constant, the other ascends for just over 1 octave.
* prompts user for file length
* files in TimbreAudio in cwd

\end{fulllineitems}

\index{wavePlot() (Timbre.timbre.Timbre method)}

\begin{fulllineitems}
\phantomsection\label{index:Timbre.timbre.Timbre.wavePlot}\pysiglinewithargsret{\bfcode{wavePlot}}{}{}
Generates a plot of the waveform for the timbre being examined

\end{fulllineitems}

\index{writeConsonantChord() (Timbre.timbre.Timbre method)}

\begin{fulllineitems}
\phantomsection\label{index:Timbre.timbre.Timbre.writeConsonantChord}\pysiglinewithargsret{\bfcode{writeConsonantChord}}{\emph{verbose=False}}{}~\begin{description}
\item[{Writes a chord with the selected half steps at the timbre's nearest consonant pitches. }] \leavevmode\begin{itemize}
\item {} 
Prompts user for length and desired half steps.

\item {} 
Has some trouble with Odd and Beam timbres

\end{itemize}

\end{description}

\end{fulllineitems}

\index{writeEqTempChord() (Timbre.timbre.Timbre method)}

\begin{fulllineitems}
\phantomsection\label{index:Timbre.timbre.Timbre.writeEqTempChord}\pysiglinewithargsret{\bfcode{writeEqTempChord}}{}{}~\begin{description}
\item[{Writes an equal tempered chord with the selected half steps. }] \leavevmode\begin{itemize}
\item {} 
Prompts user for length and desired half steps.

\end{itemize}

\end{description}

\end{fulllineitems}


\end{fulllineitems}



\chapter{Timbre.generators module}
\label{index:module-Timbre.generators}\label{index:timbre-generators-module}\index{Timbre.generators (module)}\index{Even (class in Timbre.generators)}

\begin{fulllineitems}
\phantomsection\label{index:Timbre.generators.Even}\pysiglinewithargsret{\strong{class }\code{Timbre.generators.}\bfcode{Even}}{\emph{f\_0=220}, \emph{numPartials=7}, \emph{attenuation=0.7071}}{}
Bases: {\hyperref[index:Timbre.timbre.Timbre]{\emph{\code{Timbre.timbre.Timbre}}}}

A Timbre object with Even harmonics.

\end{fulllineitems}

\index{Odd (class in Timbre.generators)}

\begin{fulllineitems}
\phantomsection\label{index:Timbre.generators.Odd}\pysiglinewithargsret{\strong{class }\code{Timbre.generators.}\bfcode{Odd}}{\emph{f\_0=220}, \emph{numPartials=7}, \emph{attenuation=0.7071}}{}
Bases: {\hyperref[index:Timbre.timbre.Timbre]{\emph{\code{Timbre.timbre.Timbre}}}}

A Timbre object with Odd harmonics

\end{fulllineitems}

\index{Evenodd (class in Timbre.generators)}

\begin{fulllineitems}
\phantomsection\label{index:Timbre.generators.Evenodd}\pysiglinewithargsret{\strong{class }\code{Timbre.generators.}\bfcode{Evenodd}}{\emph{f\_0=220}, \emph{numPartials=7}, \emph{attenuation=0.7071}}{}
Bases: {\hyperref[index:Timbre.timbre.Timbre]{\emph{\code{Timbre.timbre.Timbre}}}}

A Timbre object with Even and Odd harmonics

\end{fulllineitems}

\index{Square (class in Timbre.generators)}

\begin{fulllineitems}
\phantomsection\label{index:Timbre.generators.Square}\pysiglinewithargsret{\strong{class }\code{Timbre.generators.}\bfcode{Square}}{\emph{f\_0=220}, \emph{numPartials=7}}{}
Bases: {\hyperref[index:Timbre.timbre.Timbre]{\emph{\code{Timbre.timbre.Timbre}}}}

A Timbre object with A Square wave (additive synthesis). Odd Harmonics. Attenuation rate is fixed at 1 / n

\end{fulllineitems}

\index{Sawtooth (class in Timbre.generators)}

\begin{fulllineitems}
\phantomsection\label{index:Timbre.generators.Sawtooth}\pysiglinewithargsret{\strong{class }\code{Timbre.generators.}\bfcode{Sawtooth}}{\emph{f\_0=220}, \emph{numPartials=7}}{}
Bases: {\hyperref[index:Timbre.timbre.Timbre]{\emph{\code{Timbre.timbre.Timbre}}}}

A Timbre object with A Sawtooth wave (additive synthesis). Attenuation rate is fixed at 2*(-1\textasciicircum{}n)/n * pi, n being the partial \#

\end{fulllineitems}

\index{Triangle (class in Timbre.generators)}

\begin{fulllineitems}
\phantomsection\label{index:Timbre.generators.Triangle}\pysiglinewithargsret{\strong{class }\code{Timbre.generators.}\bfcode{Triangle}}{\emph{f\_0=220}, \emph{numPartials=7}}{}
Bases: {\hyperref[index:Timbre.timbre.Timbre]{\emph{\code{Timbre.timbre.Timbre}}}}

A Timbre object with A Triangle wave (additive synthesis). Attenuation rate is fixed at 0.8105694691387022*((-1)\textasciicircum{}((n - 1)/2.)/(n\textasciicircum{}2)) n being the partial \#

\end{fulllineitems}

\index{Custom (class in Timbre.generators)}

\begin{fulllineitems}
\phantomsection\label{index:Timbre.generators.Custom}\pysiglinewithargsret{\strong{class }\code{Timbre.generators.}\bfcode{Custom}}{\emph{freq\_array}, \emph{amps\_array}}{}
Bases: {\hyperref[index:Timbre.timbre.Timbre]{\emph{\code{Timbre.timbre.Timbre}}}}

A Timbre object with Custom harmonics.
\begin{itemize}
\item {} 
freq\_array: an array of frequencies

\item {} 
amps\_array: an array of amplitides

\item {} 
freq\_array and amps\_array must have the same length

\end{itemize}

\end{fulllineitems}

\index{Beam (class in Timbre.generators)}

\begin{fulllineitems}
\phantomsection\label{index:Timbre.generators.Beam}\pysiglinewithargsret{\strong{class }\code{Timbre.generators.}\bfcode{Beam}}{\emph{f\_0=220}, \emph{numPartials=7}, \emph{attenuation=0.7071}}{}
Bases: {\hyperref[index:Timbre.timbre.Timbre]{\emph{\code{Timbre.timbre.Timbre}}}}

A Timbre object with Beam harmonics. Partial frequencies are .4413 * f\_0*(n +.5)\textasciicircum{}2

\end{fulllineitems}



\chapter{Indices and tables}
\label{index:indices-and-tables}\begin{itemize}
\item {} 
\DUspan{xref,std,std-ref}{genindex}

\item {} 
\DUspan{xref,std,std-ref}{modindex}

\item {} 
\DUspan{xref,std,std-ref}{search}

\end{itemize}


\renewcommand{\indexname}{Python Module Index}
\begin{theindex}
\def\bigletter#1{{\Large\sffamily#1}\nopagebreak\vspace{1mm}}
\bigletter{t}
\item {\texttt{Timbre}}, \pageref{index:module-Timbre}
\item {\texttt{Timbre.generators}}, \pageref{index:module-Timbre.generators}
\item {\texttt{Timbre.timbre}}, \pageref{index:module-Timbre.timbre}
\end{theindex}

\renewcommand{\indexname}{Index}
\printindex
\end{document}
